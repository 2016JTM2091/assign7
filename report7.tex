\documentclass{article}
\usepackage[hidelinks]{hyperref}
\usepackage{filecontents}
\usepackage{graphicx}
\usepackage{wallpaper}
\usepackage{mdframed}
\usepackage{geometry}

\begin{document}
\begin{titlepage}
\begin{center}

\Huge{\textbf{PROJECT REPORT}}
\bigskip
\huge{\textbf{ Telecommunication Software Lab}}\\
\textsc{\textbf{ (ELP 718)}}\\
[2cm]



\includegraphics[scale=0.08]{iitd_logo.png}\\
[2cm]
\textbf{Submitted by:}\\


\textbf{ JYOTIRMAY MAITY}\\

\LARGE\textbf{ 2016JTM2091}\\
\Large{\textbf{ 1st SEMESTER,2016}}\\
\bigskip
\bigskip
\textbf{ASSIGNMENT : 7} \\
\bigskip
\textbf{Submission Date: 12th September, 2016}

\end{center}
\end{titlepage}

\tableofcontents
\thispagestyle{empty}
\cleardoublepage
\listoffigures
\thispagestyle{empty}
\cleardoublepage
\setcounter{page}{1}

\section{Introduction}

\Large{\textbf{ Python:}}\\
\textbf{Python} is a general-purpose interpreted, interactive, object-oriented, and high-level
programming language. It was created by Guido van Rossum during 1985- 1990.
Like Perl, Python source code is also available under the GNU General Public License(GPL).
\vspace{10mm}


Using python we can:
\begin{itemize}
\item Easy-to-learn: Python has few keywords, simple structure, and a clearly defined syntax. This allows the student to pick up the language quickly.
\item Easy-to-read: Python code is more clearly defined and visible to the eyes.
\item Easy-to-maintain: Python's source code is fairly easy-to-maintain.
\item Databases: Python provides interfaces to all major commercial databases.
\item Portable: Python can run on a wide variety of hardware platforms and has the
same interface on all platforms.
\end{itemize}
\cleardoublepage

\section{Problem Statement}
\subsection{Problem Statement 1}
In this problem we have to write a Python program that can take a big string (with spaces) as input from the command line and count number of times a word occurs in the string and also print the top 3 words in terms of their frequency of count.\\
Also print the next permutation of each word appearing in the string.\\

\subsection{Problem Statement 2}
In problem 2 task is to  design a Graphical user Interface (GUI) to depict the location of a mobile user in a square whose corner points are (1,1) (-1,1) (1,-1)(-1,-1). In real life, the user’s location would come from a database available with the MSC. For the moment, generate the user location using the random function generator function in Python to generate a number between [0,1).\\
Then check how many of user are inside the unit circle and calculate the percentage of that.

\subsection{Problem Statement 3}
Task is to design an addressing code for a shipping company that works all around India. The address given by the customer is split into fields of 
\begin{itemize}

\item Name, House No/colony/landmark
\item City
\item District
\item State/Union Territory

\end{itemize}

Let's suppose at the intake the employer enters all the above data into the computer, now the coding machine has to build two codes out of the data.

First is machine readable like barcodes, in the form 1’s and 0’s as:
IIT Roorkee  = 001
Roorkee= 010
Uttarakhand = 100
Hence the generated gives the collection center no. CCNO = 100010001

Second is human readable, build by combination of first three letters of a place.
For example :
Prof. Ram Mishra
D - 15, North Enclave
IIT Roorkee, Roorkee
Uttarakhand

Hence human readable code HCCNO =  UTTROOIIT100010001\\

Task is to create a database of some user and create functions to Add,Delete or Modify fields of the user.\\


\cleardoublepage

\section{Assumptions}
\textbf{Problem 1:}\\
In this problem assume the user should give a proper input from command line argument\\
\textbf{Problem 2:}\\
In problem statement 2, We assume user distance by creating a random number generator and calculate the percentage according to that.

\textbf{Problem 3:}\\
In problem statement 3, We assume that database is available when using the functions.
\cleardoublepage



\section{Implementation}

Implemnnt the problems using Python programming language\\

In problem 1:\\
User is given input through command line.\\
First take all the arguments and make a single string.\\
To do that I have used a list to take the string value\\
After getting the string we can count the word using inbuild function\\
To use the inbuild function we have to import Counter from collections\\
After getting the sorted word list task is to print the values.\\

\bigskip

In problem 2:\\
First we have to import random and math to generate random number and calculate square root value.\\
Define a variable to count the number of user within the circle.\\
Take input to get the number of user\\
Calculate the distance from the origin,if the value is less than 1 then increase the count variable\\
So finall we get the number of user within the circle\\
Calculate the percentage using simple equation and print the value.\\
\bigskip

In problem 3:\\
Database is generated first\\
create the functions to add,modify or delete value\\

\cleardoublepage

\section{Test Description and Result}
In problem 1:\\
After getting input from user program will calculate and print the top three word frequency on the screen\\
\bigskip
\bigskip

In problem 2:\\
After getting number of user it will calculate percentage of user within the circle and print the value on the screen\\


\bigskip
\bigskip




\cleardoublepage

\section{Screenshots}

\begin{figure}[h]

\includegraphics[scale=0.6]{prb1_7.png}
\caption{Screen shot of problem no: 1}
\end{figure}

\cleardoublepage

\begin{figure}

\includegraphics[scale=0.58]{prb2_7.png}
\caption{Screen shot of problem no: 2}
\end{figure}
\cleardoublepage

\begin{figure}

\includegraphics[scale=0.58]{github3.png}
\caption{Screen shot of tracking commit in github: 2}
\end{figure}
\cleardoublepage

\section{References and Citations}

\begin{thebibliography}{9}
\bibitem{IEEEhowto:kopka}
Mark Pilgrim, \emph{Dive into Python}, 5th~ed.\hskip 1em plus
  0.5em minus 0.4em\relax India: BPB, 2010
  
\bibitem{latex}
latex\\
 \texttt{https://www.sharelatex.com}  
 
 \bibitem{programming}
Tutorial Point\\
 \texttt{https://www.tutorialpoint.com}

\bibitem{knuthwebsite}
Google
\\\texttt{http://www.google.com}
 
  
  
  
  
\end{thebibliography}


 
\cleardoublepage


\section{Epilogue}
The creation of database and function is a bit difficult.We can use this function as like sql command.\\
problem 1 and 2 are quite understandable and easy to implement.\\



\end{document}